\documentclass{article}

\title{\textbf{Assignment 1}}
\author{Abhishek Ajit Sabnis}
\date{14 January 2022}

\begin{document}

\maketitle
\begin{center}
{\textbf{\Large Q49 }}
\end{center}

Given 30 questions in a mcq exam. Each student marks the answer randomly and independently. 


Thus, probability that A marks option a) is 1/4

\begin{equation}
    P(A=a) = 1/4
\end{equation}

\begin{equation}
    \Rightarrow P((A=a) \cup (B=a) \cup (C=a)) = \frac{1}{4} * \frac{1}{4} * \frac{1}{4}
\end{equation}

But there are 4 options that can be marked by students - a,b,c,d. Thus the probability that all students mark same option is - 
\vspace{0.5cm}

P(All students mark same for one question) = 
\begin{equation}
     = \frac{1}{4} * \frac{1}{4} * \frac{1}{4} *4 = \frac{1}{16}
\end{equation}

There are total 30 questions. Thus, the probability(p) that all students mark same for all questions is - 

\begin{equation}
     p = \frac{1}{16^{30}} = 4^{-60}
\end{equation}

\end{document}
