\documentclass{article}
\usepackage{amsmath}
\usepackage{amssymb}
\usepackage{pgfplots}
\usepackage{graphicx}
\pgfplotsset{compat=1.16}

\title{\textbf{Assignment 7 Q105 (dec 2017)}}
\author{Abhishek Ajit Sabnis}
\date{1 April 2022}

\begin{document}

\maketitle

\textbf{53) Consider a markov chain with 5 states {1,2,3,4,5} and transition matrix P = }
\begin{equation}
\begin{pmatrix}
1/2 & 0 & 0 & 1/2 & 0\\
0 & 1/7 & 0 & 0 & 6/7\\
1/5 & 1/5 & 1/5 & 1/5 & 1/5\\
1/3 & 0 & 0 & 2/3 & 0\\
0 & 5/8 & 0 & 0 & 3/8\\
\end{pmatrix}
\end{equation}

\vspace{0.3cm}

\textbf{Which of the following is true? }

\textbf{1. 3 and 1 communicating class}

\textbf{2. 1 and 4 communicating class}

\textbf{3. 4 and 2 communicating class}

\textbf{4. 2 and 5 communicating class}
\vspace{0.5cm}

\textbf{Ans} 

If i,j are states in Markov chain and i is accessible from j and j is accessible from i (written as    $i\leftrightarrow j$), then we say i and j communicate. A communication class C $\subseteq$ S is a set of states whose members communicate,i.e. i $\leftrightarrow$ j for all i, j belongs to C, and no state in C communicates with any state not in C.

From the matrix we observe clearly that we can travel from 2 to 5 and vice versa. Its also true for 1 and 4. Thus {2,5} and {1,4} are communicating classes. 


If the matrix is huge, we can solve this using Kosaraju's algorithm. It is implemented in Python.



\end{document}
